\chapter{Introduction}
\label{chapter:introduction}

The age of information has brought about a significant transformation in the way we interact with technology. Furthermore, the increase in computational needs has led to the development of more complex systems across sectors of the industry. Commonplace are interconnected heterogeneous systems that combine multiple processing elements, such as CPUs, GPUs, and FPGAs, to achieve high performance and efficiency. These systems are often used in applications ranging from data centers to edge computing, where the need for real-time processing and analysis is paramount. Despite the advancements in hardware and software, the challenge of profiling and monitoring these systems remains a critical issue. Traditional profiling techniques often fall short in providing the necessary insights into system performance, especially in heterogeneous embedded and realtime environments where multiple processing elements interact and communicate in complex ways. Consequently, the increase in computation and accelerators leads to an induced demand as developers and researchers seek to increase the workload of these systems, pushing them to their limits. Moreover the complexity of modern systems can exacerbate the challenges of profiling, as the interactions between different components can lead to unexpected performance bottlenecks and inefficiencies.

This emphasizes the need for more effective profiling techniques that can provide real-time insights into system performance, enabling developers to identify and address performance issues before they become critical. However, there remains a limit on the amount of information that can be extracted from a single self contained systems. Therefore, it would be appropriate to design a system that can extend the capabilities of traditional profiling techniques by leveraging the interconnected nature of these embedded systems and the same efficient machanisms that enable multiple processing elements to work together


% \thispagestyle{myheadings}

% \section{A few remarks before you start}
% \label{sec:history}

% Please read the short pointers below and on the subsequent pages; this will help
% you avoid frustrations when submitting the final dissertation to the library.

% Your thesis should have 1.5in left and top margins, and 1in right and bottom
% margins. Getting this right is tricky since it may depend on your particular
% Latex installation. Most likely you will need to adjust some of the dimensions
% set up at the beginning of "bu\_ece\_thesis.sty" in this folder. Basically,
% every installation should have the base margin of 1in at the left and top, but
% this is not always the case. For example, the TexStudio/MiKTeX installation this
% document was set up on, has the default top margin of 0.3125in and so an
% additional margin of 0.6875in was added via $\backslash${topmargin}. In order to
% adjust these dimensions, you may want to follow these steps:

% \begin{itemize}
% 	\item compile the document into PDF,
% 	\item open the document in Acrobat Reader, set it to full-page viewing and
% 		magnification to 100\%
% 	\item navigate to a "full" page with the text extending from the very
% 		top to the very bottom and full-width left to right,
% 	\item measure the margins and adjust accordingly,
% 	\item if you are planning to print a hardcopy, you need to make sure
% 		to select "Page scaling" to "None" in Acrobat.
% \end{itemize}

% Another issue that BU librarians may complain about and you are likely to encounter
% are long URLs or other unbreakable text. In case of long URL addresses, you
% should use the URL package; please see suitable documentation on-line.

% However, if you encounter a long unbreakable word (e.g., foreign) the URL
% package does not help. Have a look at the example extending into the page
% margin:

% \bigskip

% {\it Consider the following Java-JDT plugin name in German: "`Plugin-Entwicklungsumgebung"'.}

% \bigskip

% Clearly, this is a problem, and BU librarians will complain. One way of fixing
% this issue is to enclose the offending paragraph in {\tt
% 	$\backslash$begin\{sloppypar\}} and {\tt $\backslash$end\{sloppypar\}},
% resulting in the following outcome:

% \bigskip

% \begin{sloppypar}
% 	{\it Consider the following Java-JDT plugin name in German:
% 		"`Plugin-Entwicklungsumgebung"'.}
% \end{sloppypar}

% \bigskip

% Indeed, although the paragraph spacing becomes sloppy, at least you can hand in
% the thesis!


% LaTeX has a steep learning curve. You can use the original book by Lamport to
% learn more \cite{lamport1985:latex}, but there are many on-line resources with
% excellent instructions and examples. Just Google a LaTeX topic you would like to
% explore.

% As far as editing and compilation of LaTeX sources, if you have not found one
% yet, TexStudio seems to be quite popular.
\newpage
\section{Motivation}
%The initial effort of this work focused on bringing a low cost, low overhead memory profile that could be easily adapted to realtime embedded systems where resources are constrained and deadlines are crucial.

%also the reality is there are a lot of accelerators out there directly interfacing with the datacenters intranet 

Many hardware and software methods associated with hardware-level profiling involve a high cost of entry or significant overhead to operate. This high (either financial or compute) cost also applies to any method requiring higher information granularity. This work is geared toward addressing the gap between inflexible low-overhead hardware logic analyzers and malleable high-overhead software methods. The goal is to provide a means of analysis that can be handled remotely to alleviate the burdens on embedded systems in an effort to maintain real-time performance. This is achievable by utilizing a growing trend in COTS development boards that integrate a field-programmable gate array (FPGA) that can communicate with SoC components through high-speed buses. Furthermore, the exponential growth of networking speeds and latencies is minimizing the disparity between system buses and external communication. Coupling these two trends might allow us to expand the capabilities of a single system beyond the board by creating a link between system level mechanisms and mature networking infrastructure. 

The goal is to provide an easily extendable, accessible, and low-cost hardware profiling solution, EthHelper, that can be used in embedded systems. This paper is structured to provide abstract design considerations, ZCU102 implementation specific details, and evaluation of the prototype. We provide realistic performance metrics using synthetic and pragmatic benchmarks for the system and demonstrate the practicality of our solution. Further, we want to motivate our application of EthHelper to works of control flow integrity (CFI) checks, security threat detection, and workload analysis.

%Expand the idea of the interconnected systems and the need for more effective profiling techniques. Along with the intial quest of making an easily extendable, accessible, and low-cost hardware profiling solution that can be used in embedded systems.



% \cite{SidlerStrom} pcie fpga as a nic to provide better rdma

% \cite{AguileraVmWare} remote memory is more useful now with the networking infrastructure being fast and powerful

% \cite{SchoinasFineControlDistSharedMem} old paper about Fine-Grain Access Control for Distributed Shared Memory

% \cite{OliverIntelCCFPGA} putting fpga on coherent fabric (intel qpi) bypass pcie penalty of small packets

% \cite{MizutaniOPTWEB} networking handled by group of fpgas



% https://standards.ieee.org/wp-content/uploads/import/documents/other/eipatd-presentations/2020/D1-03-Cannon-Inter-processor-Connectivity-for-future-Centralized-Compute-Platforms.pdf  Microchip saying pcie gen 4/5 is the way to go for multi ecu systems


% \newline\newline
% vs
% \newline\newline
% The initial effort aimed to reduce the cost (e.g. monetary, overhead, implementation) of analyzing hardware-level data and providing a generic implementation for exporting bus-level memory transaction data onto an interface. Furthermore, given the modular and generic nature of the modules that were initially created, we could think beyond the limitations of exporting memory transactions and consider what kind of data to export and possible ways to use it.


\section{Related Works}
\label{sec:Related}
% There are notable works that overlap with our current methods that provide similar results or techniques.


%BBProf shares many similarities, at least in terms of the end goal, with a number of
% well-established performance analysis toolkits. The survey in [3] provides a good overview of
% popular toolkits such as Oprofile [ 7], Dprof [ 29 ], Zoom [31], DynamoRIO [ 5], Valgrind [ 27 ],
% and Pin [ 23 ]

%software profiling Valgrind, PIN,  
% Works featuring PLs in networking applications work on lowering related overheads \cite{BrunellaHxdp}\cite{SidlerStrom}\cite{MizutaniOPTWEB} through the use of FPGAs relying on PCIE bus and software to communicate with the CPU cluster. 

% flow is sw-> sw+hw -> hw -> programable logic for monitoring -> PL in networking and Remote Memory 

% The ever-growing complexities of multi-core embedded computing platforms have exacerbated the challenges of observing temporal behaviors and performance characteristics of application sets on constrained platforms. The importance of such observability lies in the verification of a system's behavior and setting expected bounds when deployed into safety-critical or realtime realms. 

Methods of system observability for developers and researchers are constantly under development and refinement, and have given us a myriad of SW and HW  implementations. 

The software space has provided multiple frameworks and solutions to address the observability of applications at runtime over the decades. \cite{ScalesShasta} is an example of the earliest attempts at providing fine-grain memory access with low overhead. \cite{MemProfSurvey} provides a general overview of common modern memory profiling toolkits such as \cite{LUK-PIN} \cite{BrueningTDI} \cite{Valgrind}. These methods employ Dynamic Binary Instrumentation (DBI) to translate and instrument on the execution of a binary on the fly. This flexibility and low manual instrumentation are coupled with the immense effort of platform-specific porting and high runtime overheads resulting from context switches for each instrumented instruction. Furthermore, memory profiling with this class of profilers requires all memory space references to be instrumented.

Other efforts in the software space aim to leverage baked-in hardware debug infrastructures to offload statistic tracking or trace capturing. Works such as \cite{RT-Bench} \cite{bellecAttackDetection} rely on Performance Monitoring Unit (PMU) \cite{ia64_swdev_vol3} \cite{aarch64_spec}, which keep track of hardware events such as retired instructions, cache and memory accesses, to profile an application at runtime with marginal overhead. \cite{chenTPA} relies on a combination of PMU and trace data by utilizing the ARM Coresight debug infrastructure \cite{CoreSight}, which exposes components such as the Trace Memory Controller (TMC)\cite{TMC} and Embedded Trace Macrocell (ETM)\cite{etm} for user configuration. This again allows one to achieve acceptable progress of an application despite contention through an added scope of observable events and statistics. The few limitations presented by this software and hardware combination are the predetermined events that can be monitored, the number of events that can be monitored concurrently, and the fetching blackout window needed to use the hardware.

This overall trend of delegating tasks to hardware for monitoring or accelerating tasks has seen efforts flourish in the programmable logic space. Similar to \cite{chenTPA}, \cite{hoppe21} uses the Coresight infrastructure to monitor applications; however, it uses an FPGA to decrease the decoding time of the debug packets, since the dataflow can be understood and optimized within configurable hardware. In a similar effort to \cite{bellecAttackDetection}, \cite{Feng21} attempts to lower the latencies of detecting attacks on control flow integrity by using the same Coresight+FPGA combination to monitor an application CFI exclusively through hardware.

The advancements in monitoring and profiling space have been improving the efficiency of singular embedded platforms by incorporating more hardware-level infrastructure to minimize latencies and overhead. Meanwhile, other spaces in the research community are seeing the applicability of remote resources \cite{AguileraVmWare}, and there have been recent efforts to adapt programmable logic to lower the overhead of these understood dataflows. \cite{MizutaniOPTWEB} proposed a fully connected network of tightly coupled FPGAs to provide significant cost reductions (e.g. packet processing time) for >100Gbps networks.~\cite{CalciuPberry} and \cite{SidlerStrom} are works that try to increase the efficiency of remote memory access through programmable logic, but in two distinct manners. \cite{SidlerStrom} is intended to expand Remote over Converged Ethernet (RoCEv2) semantics and introduce data-shuffling at the Network Interface Card (NIC) level to provide consistent and performant remote data traversal and retrieval. \cite{CalciuPberry} minimizes dirty data amplification and page faulting associated with remote memory by allowing an FPGA to track and monitor cache-coherent traffic for statistics that the host operating system can use.

The overlap of a permutation of these efforts have resulted in similar works proposes remote solutions to traditionally local operations. \cite{C-Flat} and \cite{CFA+} propose remote checking of Control Flow Attestations in order to provide a scalable and secure platform. \cite{Basile2012} explores the space of remote-code integrity verification through the incorporation of an FPGA to generate CFA and harden distributed embedded systems. Further uses of FPGA in remote verification is explored in \cite{Aysu2016}, which proposes a remote integrity verification of the physcial system in which the FPGA is embedded. 

Our work builds on the existing body of knowledge by proposing a novel approach to profiling and monitoring embedded systems through a combination of programmable logic and networking. 
%(expand on this a bit more)