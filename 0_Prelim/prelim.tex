% This file contains all the necessary setup and commands to create
% the preliminary pages according to the buthesis.sty option.

\title{AXI over Ethernet: Remote Memory Transaction Analysis via Networked Bus Traffic Forwarding}

\author{Patrick Carpanedo}

% Type of document prepared for this degree:
%   1 = Master of Science thesis,
%   2 = Doctor of Philosophy dissertation.
\degree=1

\prevdegrees{B.A., College of the Holy Cross, 2020\\}

\department{Department of Computer Science}

% Degree year is the year the diploma is expected, and defense year is
% the year the dissertation is written up and defended. Often, these
% will be the same, except for January graduation, when your defense
% will be in the fall of year X, and your graduation will be in
% January of year X+1
\defenseyear{2025}
\degreeyear{2025}

% For each reader, specify appropriate label {First, Second, Third},
% then name, and title. IMPORTANT: The title should be:
%   "Professor of Electrical and Computer Engineering",
% or similar, but it MUST NOT be:
%   Professor, Department of Electrical and Computer Engineering"
% or you will be asked to reprint and get new signatures.
% Warning: If you have more than five readers you are out of luck,
% because it will overflow to a new page. You may try to put part of
% the title in with the name.
\reader{First}{Renato Mancuso, PhD}{Associate Professor of Computer Science}
\reader{Second}{John Liagouris}{Assistant Professor of Computer Science}
\reader{Third}{Sabrina Neuman}{Assistant Professor of Computer Science}

% The Major Professor is the same as the first reader, but must be
% specified again for the abstract page. Up to 4 Major Professors
% (advisors) can be defined. 
\numadvisors=1
\majorprof{Renato Mancuso, PhD}{{Associate Professor of Computer Science\\Secondary appointment}}
% \majorprofb{First M. Last, PhD}{{Professor of Computer Science}}
%\majorprofc{First M. Last, PhD}{{Professor of Astronomy}}
%\majorprofd{First M. Last, PhD}{{Professor of Biomedical Engineering}}

%%%%%%%%%%%%%%%%%%%%%%%%%%%%%%%%%%%%%%%%%%%%%%%%%%%%%%%%%%%%%%%%  

%                       PRELIMINARY PAGES
% According to the BU guide the preliminary pages consist of:
% title, copyright (optional), approval,  acknowledgments (opt.),
% abstract, preface (opt.), Table of contents, List of tables (if
% any), List of illustrations (if any). The \tableofcontents,
% \listoffigures, and \listoftables commands can be used in the
% appropriate places. For other things like preface, do it manually
% with something like \newpage\section*{Preface}.

% This is an additional page to print a boxed-in title, author name and
% degree statement so that they are visible through the opening in BU
% covers used for reports. This makes a nicely bound copy. Uncomment only
% if you are printing a hardcopy for such covers. Leave commented out
% when producing PDF for library submission.
%\buecethesistitleboxpage

% Make the titlepage based on the above information.  If you need
% something special and can't use the standard form, you can specify
% the exact text of the titlepage yourself.  Put it in a titlepage
% environment and leave blank lines where you want vertical space.
% The spaces will be adjusted to fill the entire page.
\maketitle
\cleardoublepage

% The copyright page is blank except for the notice at the bottom. You
% must provide your name in capitals.
\copyrightpage
\cleardoublepage

% Now include the approval page based on the readers information
% Once the approval page is approved by the Mugar Library staff, please
% comment out the "\approvalpagewithcomment" line and uncomment "\approvalpage"
%\approvalpagewithcomment
\approvalpage
\cleardoublepage

% Here goes your favorite quote. This page is optional.
\newpage
%\thispagestyle{empty}
\phantom{.}
\vspace{4in}

\begin{singlespace}
\begin{quote}
  We absolutely must leave room for doubt or there is no progress and there is no learning. There is no learning without having to pose a question. And a question requires doubt. People search for certainty. But there is no certainty. People are terrified — how can you live and not know? It is not odd at all. You only think you know, as a matter of fact. And most of your actions are based on incomplete knowledge and you really don't know what it is all about, or what the purpose of the world is, or know a great deal of other things. It is possible to live and not know.
  \newline
  -Richard P. Feynman
\end{quote}
\end{singlespace}

% \vspace{0.7in}
%
% \noindent
% [The descent to Avernus is easy; the gate of Pluto stands open night
% and day; but to retrace one's steps and return to the upper air, that
% is the toil, that the difficulty.]

\cleardoublepage

% The acknowledgment page should go here. Use something like
% \newpage\section*{Acknowledgments} followed by your text.
\newpage
\section*{\centerline{Acknowledgments}}
I want to thank my advisor, Prof. Renato Mancuso, for his patience and guidance. I want to thank the people in the systems group for the helpful discussions and feedback. Special thanks to my labmates in the Cyber-Physical Systems lab for the great conversations and for putting up with my constant questions. Especially to Mattia Nicolella, Bassel El Mabsout, Francesco Ciraolo, and Dennis Hoornaert for the great conversations, help, and support in making this thesis. Also, I want to extend a special thanks to my family and friends for their support and love. Of course, I want to thank my partner, Sisary, for her love and support in this journey.


% Here go all your acknowledgments. You know, your advisor, funding agency, lab
% mates, etc., and of course your family.

% As for me, I would like to thank Jonathan Polimeni for cleaning up old LaTeX
% style files and templates so that Engineering students would not have to suffer
% typesetting dissertations in MS Word. Also, I would like to thank IDS/ISS
% group (ECE) and CV/CNS lab graduates for their contributions and tweaks to this
% scheme over the years (after many frustrations when preparing their final
% document for BU library). In particular, I would like to thank Limor Martin who
% has helped with the transition to PDF-only dissertation format (no more printing
% hardcopies -- hooray !!!)

% The stylistic and aesthetic conventions implemented in this LaTeX
% thesis/dissertation format would not have been possible without the help from
% Brendan McDermot of Mugar library and Martha Wellman of CAS.

% Finally, credit is due to Stephen Gildea for the MIT style file off which this
% current version is based, and Paolo Gaudiano for porting the MIT style to one
% compatible with BU requirements.

% \vskip 1in

% \noindent
% Janusz Konrad\\
% Professor\\
% ECE Department
\cleardoublepage

% The abstractpage environment sets up everything on the page except
% the text itself.  The title and other header material are put at the
% top of the page, and the supervisors are listed at the bottom.  A
% new page is begun both before and after.  Of course, an abstract may
% be more than one page itself.  If you need more control over the
% format of the page, you can use the abstract environment, which puts
% the word "Abstract" at the beginning and single spaces its text.

\begin{abstractpage}
% ABSTRACT

% Have you ever wondered why this is called an \emph{abstract}? Weird thing is
% that its legal to cite the abstract of a dissertation alone, apart from the
% rest of the manuscript.

% old abstract below 

Modern systems are approaching exceedingly complex designs as manufacturers are incorporating heterogeneous CPU architectures, hardware accelerators, and specialized components to address the growing amount of raw data input and the need for diverse computing resources.
However, the cost of complexity has exacerbated issues with security, power efficiency, and temporal predictability. The common denominators are a lack of understanding and limited monitoring of the complex interplay between software and hardware components. Efforts have been made to address these pitfalls by introducing methods such as software/hardware containerization, optimized schedulers, and software standards/certifications. These methods have shown considerable improvements in strengthening the security properties of complex systems, achieving power trade-offs, and mitigating key sources of temporal non-determinism. Recent trends in computing models have given ground for new techniques that increase the amount of introspection in complex systems. In particular, given the rise in popularity of programmable logic, vendors are manufacturing tightly coupled FPGAs co-located with traditional compute clusters expanding the hardware/software co-design opportunities with incredible flexibility. We postulate that this model enables remote access to metadata and data traces that are historically confined on-chip.

The proposal of this thesis is to design, implement, and evaluate proof-of-concept mechanisms with a twofold goal. First, we aim to devise a low-latency non-intrusive mechanism to monitor and/or manipulate information (e.g., about memory transactions, code execution, etc.) flowing through on-chip buses. Second, we tackle the challenge of forwarding data obtained through the first mechanism to a remote node via a dedicated high-bandwidth, low-latency interface. Appropriate packetization techniques are explored accordingly with compression given consideration. Said mechanisms enable novel paradigms for local and remote security threat identification and mitigation of edge systems. Furthermore, it could support complex remote live workload analysis with the objective of (1) achieving better power efficiency and (2) to drive local resource management and scheduling policies to achieve temporal determinism, to name a few.

\end{abstractpage}
\cleardoublepage

% Now you can include a preface. Again, use something like
% \newpage\section*{Preface} followed by your text

% Table of contents comes after preface
\tableofcontents
\cleardoublepage

% If you do not have tables, comment out the following lines
\newpage
\listoftables
\cleardoublepage

% If you have figures, uncomment the following line
\newpage
\listoffigures
\cleardoublepage

% List of Abbrevs is NOT optional (Martha Wellman likes all abbrevs listed)
\chapter*{List of Abbreviations}

%{\color{red} As per BU library instructions, the list of abbreviations must be in alphabetical order by the {\bf abbreviation}, not by the explanation, or it will be returned to you for re-ordering. {\bf This comment must be removed in the final document.}}

\begin{center}
  \begin{tabular}{lll}
    \hspace*{2em} & \hspace*{1in} & \hspace*{4.5in} \\
    ACE & \dotfill & AXI Coherency Extensions \\
    AoE & \dotfill & AXI over Ethernet \\
    APU & \dotfill & Application processor unit \\
    AXI & \dotfill & Advanced eXtensible Interface \\
    BRAM & \dotfill & Block Random Access Memory \\
    CCI & \dotfill & Cache Coherent Interconnect \\
    CLB & \dotfill & Configurable Logic Block \\
    COTS & \dotfill & Commercially-off-the-shelf \\
    CPU & \dotfill & Central Processing Unit \\
    DRAM & \dotfill & Dynamic Random Access Memory \\
    DTB & \dotfill & Device Tree Blob \\
    FF & \dotfill &  FrameFormer \\
    FFM & \dotfill & FrameFormer Manager \\
    FFS & \dotfill & FrameFormer Subordinate \\
    FPGA & \dotfill & Field Programmable Gate Array \\
    FPD & \dotfill & Full Power Domain \\
    GTH & \dotfill & Gigabit Transceiver type H \\
    ILA & \dotfill & Integrated Logic Analyzer \\
    LC & \dotfill & Lucent Connector \\
    LPD & \dotfill & Low Power Domain \\
    LUT & \dotfill & Look-Up Table \\
    MAIR & \dotfill & Memory Attribute Indirection Register \\
    NIC & \dotfill & Network Interface Card \\
    PE & \dotfill & Processing Element \\
    PL & \dotfill & Programmable Logic \\
    SFP+ & \dotfill & Small Form-factor Pluggable Plus \\
    SoC & \dotfill & System on Chip \\
    VIP & \dotfill & Verification Intellectual Property \\
    XES & \dotfill & Xilinx Ethernet Subsystem \\
  \end{tabular}
\end{center}
\cleardoublepage

% END OF THE PRELIMINARY PAGES

\newpage
\endofprelim
