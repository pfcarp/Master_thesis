\chapter{Design}
\label{chapter:design}
\thispagestyle{myheadings}
The following section provides an overview of the constraints and the requirements to achieve remote monitoring of hardware-level transactions with minimal impact. This section will provided the abstraction of \ref*{chapter: implementation_overview} and provide the design choices and backhground that influence the implementation of the system.


\section{Network Design Overview}
%The easiest thing to do as of 09/01/2024 is to broadcast the information out or send to a single node through mac. If using broadcast then the parsing nodes would be able to subscribe to this stream with a known source address. In the future I would probably adapt layer 3 protocols so I can use IP addresses.
%if not using broadcast then it becomes a server client relationship where the client is the zcu and the server is the parser

%I guess the reason I went with pub/sub is because there isnt really a feedback action in this system anyways. it just gets data and uses it but does not report anything back
In this initial work two systems are connected through a high-speed, low latency link in a producer/consumer architecture and communicate ontop a protocol that abstracts the physical layer. The producer has the capability of sending frames of bus-level data (or copies) without kernel involvement, while the consumer receives, unpacks, and parses the frames to visualize the producer's memory transaction history for observability purposes. 
%this sentence below sucks but I am trying to put together a vision where the producers are weak embedded platforms and the consumer, a luxury node, is powerful enough to handle at least not being overloaded by the stream of data. Furthermore since this project can be expanded to hand layer 3 communication, the notion of using a network switch to connect many to many or many to one is not absurd.
This system is designed to work in a many-to-one or many-to-many model, as the consumers would be magnitudes more powerful than the producers to avoid possible bottlenecks and dropped packets.
%The proposed system is intended to work with many producers and few consumers.
%expand this when you can
\subsection{Network Requirements}
The network infrastructure must adhere to stringent requirements to meet the demanding needs of tapping into producer's on-chip high-speed system bus subsystem. A minimum bandwidth of 10Gbps is acceptable, for practical purposes \ref{fig:bandwidth_result}, to sustain the rapid data exchange demanded by such subsystems. %the problem I have with this statement is that its not necessarily because of the link that performance is lower. Also, the final clock is part of the problem (data plane stuck to 156.25 MHz at the final step)
Additionally, latency must be kept below 1 $\mu$s to ensure swift communication and minimize delays in data transmission. By meeting these criteria, the network can seamlessly accommodate the large data amount generated and maintain optimal producer performance.
\subsection{Consumer Parsing}
The consumer possesses the capability to parse incoming packets efficiently, enabling the extraction of transaction details. Within these packets, the consumer can discern the nature of transactions, identify the addresses impacted by each transaction, and determine the originating system component responsible for initiating the transaction. This parsing ability equips the consumer with comprehensive insights into data flow and transactional dynamics, facilitating precise analysis
\section{Publisher Design Overview}
% \subsection{Heterogeneous Architecture}
% The Publisher system adopts a heterogeneous architecture, integrating a conventional CPU cluster(s) alongside a Programmable Logic (PL) block. This fusion of disparate structures is facilitated by a high-speed interconnect, which serves as the bridge enabling seamless communication and collaboration between the CPU cluster(s) and the PL block. By leveraging both CPU-based processing capabilities and the flexibility of programmable logic, the Publisher system achieves a versatile and powerful computing environment, capable of tackling diverse tasks and accommodating a wide range of computational requirements. Furthermore, the publisher is able to adhere to updates for data transmission as needed without the need to create new silicon or libraries.


% \subsection{Memory-Mapped System Bus}
% The assumption for any architecture is that there is some underlying hardware and scheme connecting any two elements within silicon for communication. For this paper the scheme is assumed as a memory-mapped interconnect(MMI), which dictates that all components (e.g. cpu cluster, PL block, accelerators, etc) are giving a physical address and a scope. Communication is handled by the shared bus directing data and signals to the appropriate address and the corresponding device should handle the data according to a protocol.
% %what to say after this? easier use and ease of implementing something ontop of a mm-interconnect?

The publisher system needs to provide a few basic components and properties. The system must have a processing element (PE), programmable logic (PL), external interface for network communication, main memory, an Observer within the PL, and a system bus subsystem to tie everything together. All components can communicate with any other components independently through the interconnect or other dedicated internal links. This communication must be standardized on all components and should use address space for directing access to the appropriate components, memory, or I/O devices. Communication with the bus and components is generally assumed to be parallel in nature (e.g. reads and writes can occur at the same time given no conflict). Considering these requirements, we will be focusing on specific applications of bus and external communication going forward. Therefore, we will implement \MethodNameLong (\MethodNameShort) in the following manners seen in \ref*{fig:all_paths}.




% Considering these requirements, the \MethodNameLong ~(\MethodNameShort) can be implemented in the following manners seen in \ref*{fig:all_paths} explained in the following portion.

%I think I should take more about the PE/PL boundary and the problems caused by it
%expand these a bit more. I think I can really milk it if I wanted too
First, transactions on the system bus are explicitly sent to the PL for duplication~\ref{fig:paths_duplicated}. This provides the shortest critical path for explicit routing given that duplication does not imply blocking caused by the Observer's mechanisms. However, this a fundamentally lossy tracking architecture which may unpredictably drop data at the convenience of not slowing PE.

Second, transactions on the system bus are explicitly routed through the Observer in the PL~\ref{fig:obs_on_the_path}. This allows for lossless tracking of transactions albeit at the cost of having all of the mechanisms involving the external interface on the critical path.

Lastly, cache-coherent system buses broadcast transactions which the Observer can passively tie into~\ref{fig:paths_snoop}. A schema that would allow the Observer mechanisms to function without explicit routing to the PL address space. Furthermore, it comes at minimal cost since the transaction is not explicitly crossing the PE/PL boundary. 

For ease of implementation, we will be implementing any modules to follow the~\ref{fig:obs_on_the_path} schema. This will provide a lossless tracking mechanism with minimal engineering overhead. 

%add more labels to each figure designating PE or PL 
%the captions are megafucked and IDK why
\begin{figure}
    \begin{minipage}[b]{0.32\textwidth}
        \input{imgs/Generic_Duplicated}
        \caption{Transactions explicitly routed and duplicated within the PL}
        \label{fig:paths_duplicated}
    \end{minipage}
    \begin{minipage}[b]{0.32\textwidth}
        
        \input{imgs/Generic_on_the_Path}
        \caption{Transactions are explicitly routed through the Observer within the PL}
        \label{fig:obs_on_the_path}
    \end{minipage}
    \begin{minipage}[b]{0.32\textwidth}
        \input{imgs/Generic_Snoop}
        \caption{Transactions are broadcast on the System Bus with a passive Observer}
        \label{fig:paths_snoop}
    \end{minipage}
    \caption{Fundamental Routing Schemas}
    \label{fig:all_paths}
\end{figure}

% \begin{figure}
%     \centering
%     % First minipage
%     \begin{minipage}[b]{0.3\textwidth}
%         \centering
%         \input{imgs/Generic_Duplicated}
%         \parbox{1.5\linewidth}{\centering\caption*{Transactions explicitly routed and duplicated within the PL}}
%         \label{fig:paths_duplicated}
%     \end{minipage}
%     \hfill
%     % Second minipage
%     \begin{minipage}[b]{0.3\textwidth}
%         \centering
%         \input{imgs/Generic_on_the_Path}
%         \parbox{1.5\linewidth}{\centering\caption*{Transactions are explicitly routed through the Observer within the PL}}
%         \label{fig:paths_on_the_path}
%     \end{minipage}
%     \hfill
%     % Third minipage
%     \begin{minipage}[b]{0.3\textwidth}
%         \centering
%         \input{imgs/Generic_Snoop}
%         \parbox{1.5\linewidth}{\centering\caption*{Transactions are broadcast on the System Bus with a passive Observer}}
%         \label{fig:paths_snoop}
%     \end{minipage}
%     \caption{Overall caption for the three subfigures.}
%     \label{fig:all_paths}
% \end{figure}
\section{PL Design Overview}
\subsection{Memory-mapped Programmable Logic Block}
%FPGA has multiple ports that are all tied to a memory address for ps-pl communication. this allows for a flexible design pattern 
We assume that the PL block has multiple address spaces that can be used to account for possible different power/time domains on a system. These physical address spaces can be allocated to the same or separate physical ports on the Bus. This allows for the PL to contain multiple mechanisms to generate traffic on separate ports to the Bus.
\subsection{Dedicated hardware}
%Besides LUTS for constructing logic, the FPGA has dedicated hardware to accelerate or augment its processing power. In some case externally linked devices are linked to the fpga
The basic building blocks of FPGAs are Look Up Tables (LUTs) which can be composed into fundamental transistor logic (e.g. AND, OR, etc). Throughout the development of the FPGA dedicated hardware blocks have been embedded into the FPGA to accelerate computation or introduce new functionality. For this paper, we will focus on the use of three dedicated hardware blocks found in modern FPGAs. Digital Signal Processors (DSPs) are computational accelerators for multiplication and division. Block RAM was invented to bring memory inside the FPGAs. Finally, GTH/GTY transceivers were introduced to handle external communication. 
%using the fpga to look on the memory mapped interconnects can allow for a snoop to happen without the OS intervening. When snooping the data can be stored and sent to another location all through fpga mechanism

%I think the section below should have a diagram of the general mechanisms of the Observer
\subsection{Observer Mechanism}
The \emph{Observer} has to fulfill several tasks: (1) communicate with the system bus, (2) form packets of transaction data, (3) instantiate a controller to use the external interface, and (4) send the data through the physical port. We show a generic module and layout to achieve the route through schema ~\ref{fig:obs_on_the_path} with ~\ref{fig:PL_Module}. 


%I have not mentioned the parallel nature of the system bus anywhere
\emph{Serializer/Communicator} is the initial point of the \emph{Observer} mechanisms which allows communication to and from the System Bus. Furthermore, This module is responsible for serializing data and metadata of R/W requests and distributing the data stream among available external interfaces. 
%zcu102 has ddr4 2666v which has a peak transfer rate of 21.3333GB/s and a CAS Latency of 14.25 ns
%actually the default config at the time of testing was DDR4-2133P with 17.06667GB/s and 14.063ns with the said ddr4-2666v so-DIMM module
%given the above the memclk is 266.67MHz ioclk 1066.67MHz on the table

%delays 
%orch handshake+schedule = 3 clk cycles
%Ethernet tready blackout = 2 clk cycles
%


The \emph{Packet Former Module (PFM)} is responsible for assembling serialized data and interfacing with the external interface controller. The PFM fulfills its responsibility by accumulating and packaging bus data to comply with the external interface protocols. In addition, this module abstracts control signals between the Serializer/Communicator and the \emph{Medium Controller} to allow for the accumulation of data without blocking the PE and correct communication. 

A \emph{Medium Controller} needs to be instantiated to ingest packets from the Packet Former and manage external interface transceivers to transmit data. Moreover, this module offers debug signals to verify protocol compliance and transceiver operation. The medium and protocol will dictate what component in the PL will contribute to the overall additional latency of using \MethodNameShort. In respect to the minimum requirements, the latency and throughput of the system will be limited by the medium as DDR4 (the defacto memory in use) has a sustained raw throughput of 12.8 GB/s with 15ns CAS latency~\ref{fig:ddr4specs}.

\begin{figure}
\input{imgs/Generic_PL_Layout}
        \caption{General module implementation in PL. For now, Communicator has Bidirectional communication only with non-PL components}
        \label{fig:PL_Module}
\end{figure}

% Using the PL we can instantiate a mechanism to route transactions occurring on an Bus and have that information sent off-board through a transceiver. This requires custom hardware to manage the interactions between a Memory-mapped piece of memory and an interconnect so that the transactions do not overwhelm the transceiver's capabilities which would cause dropped packets, malformed packets, etc. 
% %explain more how it works and what individual parts may exist because of such proposed methods
% As this process happens on the hardware level and its components are isolated in the FPGA, the OS is unaware of the underlying mechanism.

