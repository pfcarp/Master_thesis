\chapter{Additional Results}
\label{appendix}
\thispagestyle{myheadings}



\begin{table}[t]
    \centering
    \resizebox{.99\columnwidth}{!}{
        \begin{tabular}{cccc}
        \textbf{Program} & \textbf{normal dram} & \textbf{no ETH} & \textbf{ETH} \\
        \toprule
        \textbf{disparity} & 8.62e-04 $\pm$ 2.30e-05 & 8.61e-04 $\pm$ 3.60e-05 & 8.64e-04 $\pm$ 4.40e-05\\
\midrule
\textbf{mser} & 1.46e-03 $\pm$ 2.20e-05 & 1.45e-03 $\pm$ 3.00e-05 & 1.47e-03 $\pm$ 3.70e-05\\
\midrule
\textbf{sift} & 2.66e-03 $\pm$ 2.30e-05 & 2.64e-03 $\pm$ 2.20e-05 & 2.65e-03 $\pm$ 3.00e-05\\
\midrule
\textbf{stitch} & 3.55e-04 $\pm$ 1.60e-05 & 3.47e-04 $\pm$ 1.80e-05 & 3.47e-04 $\pm$ 1.80e-05\\
\midrule
\textbf{texture\_synthesis} & 9.91e-04 $\pm$ 3.10e-05 & 9.90e-04 $\pm$ 3.20e-05 & 1.00e-03 $\pm$ 3.20e-05\\
\midrule
\textbf{tracking} & 2.00e-03 $\pm$ 2.00e-05 & 1.97e-03 $\pm$ 2.10e-05 & 1.97e-03 $\pm$ 2.40e-05\\

\bottomrule \end{tabular}}
\caption{Runtime comparison between different memory routings for sim\_fast image size.}
\label{tab:SD_VBS-results-sim-fast-no-contention}
\figvspace
\end{table}



\begin{table}[t]
    \centering
    \resizebox{.99\columnwidth}{!}{
        \begin{tabular}{cccc}
        \textbf{Program} & \textbf{normal dram} & \textbf{no ETH} & \textbf{ETH} \\
        \toprule
        \textbf{disparity} & 1.94e-03 $\pm$ 2.50e-05 & 1.95e-03 $\pm$ 3.70e-05 & 1.95e-03 $\pm$ 4.30e-05\\
\midrule
\textbf{mser} & 4.89e-03 $\pm$ 4.20e-05 & 4.90e-03 $\pm$ 1.28e-04 & 4.91e-03 $\pm$ 1.52e-04\\
\midrule
\textbf{sift} & 1.37e-02 $\pm$ 4.90e-05 & 1.36e-02 $\pm$ 7.40e-05 & 1.37e-02 $\pm$ 7.10e-04\\
\midrule
\textbf{stitch} & 5.96e-04 $\pm$ 1.80e-05 & 5.86e-04 $\pm$ 1.80e-05 & 5.85e-04 $\pm$ 1.80e-05\\
\midrule
\textbf{texture\_synthesis} & 6.67e-03 $\pm$ 4.37e-04 & 6.61e-03 $\pm$ 1.02e-04 & 6.62e-03 $\pm$ 1.04e-04\\
\midrule
\textbf{tracking} & 5.27e-03 $\pm$ 2.90e-05 & 5.19e-03 $\pm$ 3.30e-05 & 5.18e-03 $\pm$ 2.80e-05\\

\bottomrule \end{tabular}}
\caption{Runtime comparison between different memory routings for sim image size.}
\label{tab:SD-VBS-results-sim-no-contention}
\figvspace
\end{table}


\begin{figure}%
    \centering
    \subfloat[\centering without interference]{{\includegraphics[width=\linewidth]{imgs/tracking_scatterplot_no_interference.png} }}%
    \qquad
    \subfloat[\centering with interference]{{\includegraphics[width=\linewidth]{imgs/tracking_scatterplot_3_core_interference.png} }}%
    \caption{tracking with sim image size}%
    \label{fig:tracking}%
  \end{figure}

  \begin{figure}%
    \centering
    \subfloat[\centering without interference]{{\includegraphics[width=\linewidth]{imgs/mser_scatterplot_3_core_interference.png} }}%
    \qquad
    \subfloat[\centering with interference]{{\includegraphics[width=\linewidth]{imgs/mser_scatterplot_3_core_interference.png} }}%
    \caption{mser with sim image size}%
    \label{fig:mser}%
  \end{figure}

\begin{figure}
  \includegraphics[angle=90,width=\linewidth,height=20cm]{imgs/Tested_Debug_Design.png}
  \caption{Entire PL Design used for Debugging and Testing}
  \label{fig:entire_test_pl_design}
\end{figure}

% \section{Code Snippets}
% This section contains code snippets from the implementation of key components in the system.

% \subsection{Orchestrator Control Logic}
% The following code snippet shows the control logic of the Orchestrator that manages state transitions and fair scheduling between submodules:

% \begin{lstlisting}[language=Verilog, title=Orchestrator Control Logic]
% genvar k;
%   generate
%    for (k = 0; k < channels; k++) begin
%       //Make first encoding all 0 for NONE state
%       if(k>0)begin
%         assign encodings[k] = {{channels{1'b0}} | {1'b1}}<<k;
%       end
%       else begin
%         assign encodings[k] = {channels{1'b0}};
        
%       end
%    end
%    endgenerate

%         assign submodule_data[0]=0;
%         assign submodule_in_progress[0]=0;
%         assign submodule_last[0]=0;
%         assign submodule_valid[0]=0;
%         assign resets[0]=1;
%         assign submodule_transaction_length[0]=0;
%         //assign submodule_resets[0]=0;

%   reg [5:0] transaction_length;
%   //stateChanger!
%   integer i=0;
%   always @(posedge clk) begin
%     if (!resetn) begin
%     //last_index<=0;
%     State<=0;
%     timestamp<=0;
%     end
%     else begin
%       //if the state that we were in last is still performing a transaction and/or downstream is not ready then do not change states 
%       if (~submodule_in_progress[State] & transaction_length==0)  begin
%         //check if any submodules have data pending
%         if(|submodule_valid)begin
%           //change state to the next submodule that has valid data waiting to be sent
%           for (i = 1; i <=channels; i++) begin 
%             //explicitly check to wrap around when checking who is the first valid submodule
%             if (submodule_valid[(State+i>channels) ? (State+i-channels):(State+i)])begin
%                //change the state when the first valid submodule is found
%                State<=(State+i>channels) ? (State+i-channels):(State+i);
%                break;
%             end
%           end
%         end
%         //if no submodule ready, then switch to the NONE State
%         else begin
%           State<=0;
%         end
%       end
%       //state is in progress. update something 
%       else begin
%         State<=State;
%       end
%       timestamp<=timestamp+1;
%     end
%   end
% \end{lstlisting}

% \subsection{Orchestrator Dataflow Logic}
% The following code snippet shows the dataflow logic of the Orchestrator that handles the \axistream~ interface and submodule orchestration:

% \begin{lstlisting}[language=Verilog, title=Orchestrator Dataflow Logic]
%  reg started;
%   //DATAMOVER!
%   int j;
%   //this block will handle AXISTREAM and Submodule Orchestration
%   always @(posedge clk) begin
%     if(!resetn)begin
%       Orchestrator_ready<=encodings[0];
%       stream_tdata<=0;
%       stream_tlast<=0;
%       stream_tvalid<=0;
%       transaction_length<=0;
%       started<=0;
%     end
%     else begin
%       if(State!=0 & stream_tready)begin
%         if(transaction_length>=1 & started)begin
%           //start the transaction
%           Orchestrator_ready<=encodings[State];                             
%           stream_tdata<=submodule_data[State];
%           stream_tlast<=submodule_last[State];
%           stream_tvalid<=1;
%           transaction_length<=transaction_length-1;
%         end
%         else if (transaction_length==0 & started) begin
%           Orchestrator_ready<=encodings[0];
%           stream_tdata<=0;
%           stream_tlast<=0;
%           stream_tvalid<=0;
%           transaction_length<=0;
%           started<=0;
%         end
%         else if (!started & submodule_valid[State])begin
%           //grab the length
%           transaction_length<=submodule_transaction_length[State];
%           started<=1;
%         end
        
%       end
%       else begin
%         Orchestrator_ready<=encodings[0];
%         stream_tdata<=0;
%         stream_tlast<=0;
%         stream_tvalid<=0;
%         transaction_length<=0;
%         started<=0;
%       end
%     end
%   end
% \end{lstlisting}

% \subsection{FrameFormer Subordinate Logic}
% The following code snippet shows the core functionality of the FrameFormer Subordinate (FFS) that provides a shifting register to buffer incoming data:

% \begin{lstlisting}[language=Verilog, title=FrameFormer Subordinate Logic]
%     always @ (posedge ACLK)begin
%         if(!ARESETN)begin
%             tail <= 0;
%             for(i=0;i<MAX_INTERNAL_SPACE;i=i+1)begin
%                 tempReg[i]<={((OUTPUT_WIDTH)){1'b0}};
%             end
%             delayedOut<=0;
%             counterPulseOutReg<=0;

%         end
%         else begin

%             if(InputCondition & !OutputCondition)begin
%                 tempReg[tail]<=S_AXIS_tdata;
%                 tail<=tail+1;
%                 counterPulseOutReg<=0;
%             end
%             else if(OutputCondition & !InputCondition)begin
%                 shiftandOutput();
%                 if (tail!=0)
%                     tail<=tail-1;
%                 delayedOut<=tempOut;
%                 counterPulseOutReg<=1;               
%             end
%             else if(OutputCondition & InputCondition)begin
%                 shiftOutputInsert();
%                 if(empty)
%                     tail<=tail+1;
%                 else
%                     tail<=tail;
%                 delayedOut<=tempOut;
%                 counterPulseOutReg<=1;
%             end
%             else begin
%                 for(i=0;i<MAX_INTERNAL_SPACE;i=i+1)begin //make every register 0 to avoid a floating register
%                     tempReg[i]<=tempReg[i];
%                 end
%                 delayedOut<=tempOut;
%                 counterPulseOutReg<=0;
%             end
%         end 
        
%     end
% \end{lstlisting}

% \subsection{FrameFormer Manager Logic}
% The following code snippet shows the core functionality of the FrameFormer Manager (FFM) that handles the creation and sizing of ethernet frames:

% \begin{lstlisting}[language=Verilog, title=FrameFormer Manager Logic]
% always @ (posedge ACLK)begin
%         if(!ARESETN)begin
%             init();
%         end
%         else if(M_AXIS_tready || state==0)begin
%             if(state==Packet_Size)begin//last
%                 M_AXIS_tlast<=1;
%                 state<=state+1;
%             end
%             else if (state>Packet_Size) begin
%                 M_AXIS_tvalid<=0;
%                 M_AXIS_tlast<=0;
%                 state<=0;
%             end
%             else if (state!=0 )begin
%                 M_AXIS_tvalid<=1;
%                 state<=state+1;
%             end
%             else if (state==0 & !is_empty)begin
%                 state<=state+1;
%                 M_AXIS_tvalid<=1;
%             end
%         end
%         else
%             state <= state;
% end
   
%    always @ (posedge ACLK) begin
%         if(!ARESETN)begin
%             DataInit();
%             counterPulseOutReg<=0;
%         end
%         else if(M_AXIS_tready || state==0)begin
%             if(state==Packet_Size)begin 
%                 M_AXIS_tdata <= 64'h5704;
%                 M_AXIS_tkeep <= 8'h07;
%                 counterPulseOutReg<=0;
%             end            
%             else if (state==0 || state>Packet_Size)begin
%                 M_AXIS_tdata[63:48] <= Source_Address[15:0];
%                 M_AXIS_tdata[47:0] <= Destination_Address;
%                 M_AXIS_tkeep <= 8'hFF;
%                 counterPulseOutReg<=0;
%                 end
%             else if (state==1)begin
%                 M_AXIS_tdata[63:48] <= SyncWord;
%                 M_AXIS_tdata[47:32] <= Link_Type;
%                 M_AXIS_tdata[31:0] <= Source_Address[47:16];
%                 M_AXIS_tkeep <= 8'hFF;
%                 counterPulseOutReg<=0;
%                 end
%             else if (state>1 & state<Packet_Size)begin
%                 M_AXIS_tdata <= Input_Data;
%                 M_AXIS_tkeep <= 8'hFF;
%                 counterPulseOutReg<=1;

%             end

%         end
%         else begin
%             M_AXIS_tdata <= M_AXIS_tdata;
%             counterPulseOutReg<=0;
            
%         end
%     end
% \end{lstlisting}


\begin{figure}%
  \centering
  \includegraphics[scale=0.75]{imgs/ddr4specs.png}
  \caption{DDR4 Memory speeds and latencies}%
  \label{fig:ddr4specs}%
\end{figure}

