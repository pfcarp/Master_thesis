\chapter{Implementation Overview}
\label{chapter: implementation_overview}
\section{System Overview}
%talk breifly about the two computers and their specs
%describe how they will be connected and tested
Our test environment contains two systems that serve the purpose of publisher and subscriber. Our subscriber will be a Dell Precision 7950 equipped with Dual Intel Xeon Gold 6130 and a Mellanox ConnectX-2 Dual SFP+ NIC running Ubuntu 22.04. The publisher is a Xilinx Ultrascale+ ZCU102 with ZU9EG SOC running a generic Petalinux with SFP+ transceivers enabled. The FPGA bitstream was compiled with Vivado v2019.2. These two systems communicate through a single point-to-point connection via 10GBase-SR SFP+ LC Transceiver and OM3 fiber cable. We used tcpdump, a cli tool, to save the exported ethernet packets 

%\section{Subscriber Details} any details about the subscriber are really just hurdles found during evaluation

\section{ZCU102 Overview}
The ZCU102 is an embedded development platform commercially-of-the-shelf (COTS) developed by AMD (previously Xilinx) equipped with ZU9EG SOC. 
This SOC is a heterogeneous architecture with an application processor unit (APU) composed of quad ARM Cortex-A53, dual Cortex-R5, and an UltraScale+ FPGA. 
These three main components interact through three communication modules: the Cache Coherent Interconnect (CCI), the Full Power Domain (FPD) main switch, and the Low Power Domain (LPD) main switch \ref{fig:Internal_Organization}. 
The scope of this paper will encompass the APU, FPGA with dedicated SFP+ transceivers, and FPD main switch. 
All memory allocated for applications will be routed through the FPGA aperture and ultimately be in system memory. 
This initial concept is possible through reserving memory in the Device Tree Blob (DTB), RT-Bench, and the Advanced eXtensible Interface (AXI) protocol.



\begin{figure}
    \centering
    \includesvg[scale=0.5]{imgs/SystemDiagramSimple.drawio.svg}
    \caption{Internal Layout of the ZCU102 and EthHelper components}
    \label{fig:Internal_Organization}
\end{figure}
    

%The ZCU102 board has numerous external hardware for communication but we will be using the SFP+ cages dedicated to the FPGA. 

%For the sake of this paper, we will be using a system ram  in the FPGA and communicate through the Main Switch and keep the SFP+ transceivers running at 10Gb/s instead of the maximum of 25Gb/s.
%The methods of this paper can be applied to either bus however we will focus on communication through the Main Switch.
\subsection{Advanced eXtensible Interface}
Heterogeneous systems necessitate a common form of communication to facilitate computation and transactions between components of differing underlying mechanisms. 
Our choice of platform is constructed around the AXI protocol for most inter-chip communication. 
AXI Coherency Extensions (ACE) are used exclusively for high-speed components connected directly to the CCI.However, ACE is a bidirectional protocol for maintaing coherence with APU caches. ACE is also limited as only read type transactions can be seen excluding any write backs without explicit routing.
This will be out of scope for the paper but is not a limitation of the current work or platform.
AXI (and ACE by extension) is a memory-mapped communication protocol that relies on (1) manager and subordinate scheme and (2) a handshake mechanism to execute memory transactions.
AXI comes in three variations FULL, LITE, and STREAM. This paper will touch directly on \axifull~ and \axistream~ as these are the necessary components for Transparent Snooping.
\axifull~ is a bidirectional protocol that employs 5 channels (2 for reading memory and 3 for writing) that operate independently and in parallel. 
In contrast, \axistream~ is a single-channel unidirectional protocol intended for data streams.

\section{FPGA Overview}
%job of the fpga is to copy data of 5 channels to 1 channel of the same speed to dedicated GTH transciever to, eventually, the Subscriber machine
%Here I talk about the orchestrator mechanisms and the BRAM leading to the Frame Former to the Eth Subsystem

%There are two major tasks that the FPGA has to achieve. First, It has to serialize the \axifull~ channels to a single \axistream~ channel.
The FPGA has 4 major tasks: redirecting memory transactions from APU to DDR memory, serializing of \axifull~ to \axistream~, forming an ethernet frame, and correctly using the ethernet module to send packets. We created the Orchestrator module and the FrameFormer module to utilize the Xilinx-provided 10g/25g ethernet subsystem (XES) module. We shall refer to all modules collectively as the EthHelper.

The Orchestrator imposes a simple fair scheduler to schedule up to 5 submodules to serialize the \axifull~ channels (e.g. AR, AW, R, W, B) to a single \axistream~ channel. Furthermore, the Orchestrator was architected to be modular, allowing the creation of submodules that address other protocols that implement handshaking. For a proof of concept, we constructed the AR and AW submodules to give us metadata consisting of transaction type, AXI ID, Burst Length, and a clock-based timestamp. The other channels are connected to passthrough dummy modules that allow uninterrupted execution, yet the scheduler is implemented for all 5 submodules to work. The current iteration of the Orchestrator requires a source and target that communicates through \axifull~ to generate understandable \axistream~ outputs.

The FrameFormer (FF) is a custom FIFO module that allows configurable ethernet packet arguments (except for 802.1Q tag and CRC) to frame and send incoming \axistream~ data. The FF is internally split into two parts: the FrameFormerSubordinate (FFS) and the FrameFormerManager (FFM). The FFS provides a shifting register to buffer and send all incoming data for the FFM. the FFM provides the control outputs for the XES and will create and size the ethernet frame with FFS data and inputs given. The mechanism is as follows: FFS will wait for a single \axistream~ burst to write into the shifting register, upon receiving the FFS will initiate the FFM to output all parts except payload to the XES, after the FFM will receive all data within the FFS shifting register until packet size is reached, once the packet size is reached it will either idle or redo the cycle depending if there is still more data.


% The Orchestrator imposes a simple fair scheduler to schedule generic submodules to serialize the AXIFULL channels (e.g. AR,AW.R.W.B) to a single AXIStream channel in a 
% transparent matter (i.e. Transparent Snooping). For the proof of concept we constructed the AR and AW submodules to give us metadata (e.g. transaction type, AXI ID, Burst Length, and a timestamp) while the other channels are passthrough dummy modules that allow uninterrupted execution. 



\chapter{Implementation Details}
%should I also mention the details of the link medium e.g. optical cables and the sfp+ transceivers?

%quick question to myself: I Know that the dac did not work in the very start... but how exactly did it not work?????
\section{Subscriber In-depth view}
\subsection{Memory and Buffers}
%nic buffer, kernel buffer timers, kernel buffer, tcpdump
% The subscriber should maximize all buffers and timers associated with packet analysis. 


% There are two types of Buffers: the Hardware NIC Buffer (one for rx and tx) and buffer in the kernel. these buffers are necessary to bridge the potential difference in network speed and to align the data flow rate with the processing capabilities of the device, preventing loss of data and improving overall transmission efficiency. A larger buffer on the NIC allows for fewer interrupts going to the CPU and allows for jumbo frames (packets over 1500 Bytes defined by IEEE 802.3) which decrease protocol overheads. this may affect network latency however this work only deals with receiving and profiling. 

% The kernel buffer allows the NIC to empty its buffer to receive more incoming packets. With a Kernel buffer too small the NIC would either overwrite the data on the NIC Buffer (NAPI) or thrashing (netif_rx()).
% %the reason why packets drop on the nic I think is because of NAPI (read advantages) https://en.wikipedia.org/wiki/New_API
% Importantly, the Network Interface Card (NIC) RX buffer size should be verified as the defaults are 256-1024 Bytes in contrast to the common maximum on 10Gbe NICs of 8192 Bytes. The Kernel Buffer should also be proportionately bigger. 
%https://wiki.linuxfoundation.org/networking/kernel_flow

Network data transmission relies heavily on buffering mechanisms to reconcile the asynchronous nature of network interfaces and host system processing capabilities. Two primary buffer types are employed: hardware-managed buffers within the Network Interface Card (NIC) and kernel-space buffers. These buffers serve a critical role in mitigating the performance disparity between network link speeds and the processing rate of the host system, preventing data loss and optimizing overall RX/TX efficiency. The hardware NIC buffer, typically one for transmit (TX) and one for receive (RX) operations, provides a temporary storage area for data awaiting processing. Increasing the size of the NIC RX buffer can significantly reduce the interrupt load on the host CPU, as fewer interrupts are required to signal per given packet amount. Furthermore, larger RX buffers enable the transmission and reception of jumbo frames (packets exceeding the standard IEEE 802.3 Ethernet frame size of 1500 bytes), which can reduce protocol overhead by consolidating multiple smaller packets into a single, larger frame. While increasing buffer sizes can improve throughput, it is important to acknowledge the potential impact on network latency, a factor not directly addressed within the scope of this research.

The kernel buffer acts as an intermediary, facilitating the transfer of data from the NIC hardware buffer to the user application memory. Insufficient kernel buffer capacity can lead to two primary issues: NIC buffer overflow (occurs in Linux 2.6+ with New API (NAPI) implementation), where the NIC hardware buffer overwrites incoming data, or excessive kernel buffer thrashing, manifested as frequent calls to netif\_rx() for pre-NAPI implementations \ref*{fig:Network_Scheme_pre_2.6}, which both are indicative of a bottleneck.  The kernel buffer's size must be carefully considered in relation to the NIC's RX buffer size to ensure efficient data flow and prevent these detrimental effects.

A significant performance consideration is the often default configuration of NIC buffer sizes.  Many network interfaces are configured with RX buffer sizes ranging from 256 to 1024 bytes, significantly smaller than the NIC's maximum buffer size of 4096-8192 bytes found on modern Ethernet NICs. Consequently, the kernel buffer must be proportionally sized to accommodate the increased data volume handled by the larger NIC RX buffer, ensuring a balanced and efficient data transfer pipeline.

        \begin{figure}
            \centering
            \includegraphics[]{imgs/Linux-Network-Stack-RX-scheme-in-kernels-PlaceHolder.png}
            \caption{Buffer usage in linux pre 2.6.}
            \label{fig:Network_Scheme_pre_2.6}
        \end{figure}
        %add figure for post linux 2.6
        
\subsection{Parsing and visualization}
% %tcpdump to parser (c++ -> python)
% The software stack relies on tcpdump (a cli packet analyzer), two C++ programs for formatting and parsing the packets, and a python script to interpret and visualize the parsed data. Tcpdump happens during runtime of the program and the rest are executed offline (with respect to the program being monitored).


% % ADDRESS 0000000870000008
% % METADATA 40000035a637eaaa
% % stream_type: 4
% % axid: 0000035a
 
%  The software stack relies on tcpdump (a cli packet analyzer), two C++ programs for formatting and parsing the packets, and a python script to interpret and visualize the parsed data. Tcpdump happens during runtime of the program and the rest are executed offline (with respect to the program being monitored). 

% Traces from tcpdump are preprocessed by packetStripper.cpp to extract the raw hexadecimal data. Then packet_processor.cpp identifies identifying start and end phrases (user defined) of the FPGA data, removes padding, byte flips the data and arranges the data in a human-readable manner displaying address, metadata, and the decomposition of metadata (Axid, AXI burst length, transaction type, and clock cycle timestamp). Each one of the steps is saved into its separate file for ease 
% % burst_length: 6
% % timestamp: 37eaaa

% Traces from tcpdump are preprocessed by packetStripper.cpp to extract the raw hexadecimal data. Then packet\_processor.cpp identifies identifying start and end phrases (user defined) of the FPGA data, removes padding, byte flips the data and arranges the data in a human-readable manner displaying address, metadata, and the decomposition of metadata (Axid, AXI burst length, transaction type, and clock cycle timestamp). Each one of the steps is saved into its separate file for ease of debugging and manipulation. After the processing of the trace, Mapper.py takes the arranged data and constructs two scatterplots that show transaction types throughout time on an address range (either at byte or page granularity).

The data acquisition and analysis pipeline constructed for this research utilizes a combination of command-line tools and custom software programs. Packet capture is performed using tcpdump, a well-known command line packet analyzer, during the runtime of the publisher system application under observation. The subsequent processing and visualization of the data are executed offline. Furthermore, the process is not limited to this operation scheme.

The proccessing of packets of the pipeline is handle by two C++ parsers: packetStripper.cpp and packet\_processor.cpp.The process starts with the preproccessing of raw packets captured by tcpdump  with packetStripper.cpp. This step extracts the raw hexadecimal data of the captured packet trace. Then packet\_processor.cpp identifies user-defined start and end delimiters within the FPGA data stream. This design choice of start and end delimiters allow identifying a complete and fully formed packet. After, packet\_proccessor.cpp performs several transformations: removal of inter-transaction padding, byte order inversion, and reformatting of the data into a human-readable structure. This structured data includes address information, metadata, and a decomposition of the metadata into fields such as AXI ID, AXI burst length, transaction type, and a clock cycle timestamp. To facilitate debugging and modularity, each processing stage is output to a separate file.

% \todo[inline]{include a portion of both cpp code snippets to show the data flow and processing steps.}



The final stage of the pipeline is implemented in Mapper.py, a Python script responsible for constructing the visualization of the processed data. This script uses the parsed and formatted data and generates two scatterplots. These scatterplots depict the distribution over time of transaction types across a defined address range, with the granularity of the address at byte-level and page-level.

\begin{figure}
            \centering
            \includegraphics[scale=0.4]{imgs/scatterplot_write_200B_10iter.png}
            \includegraphics[scale=0.4]{imgs/pages_scatterplot_write_200B_10iter.png}
            \caption{Example of a FPGA trace visualized with data obtained by the subscriber}
            \label{fig:example_trace_visualized}
\end{figure}
\section{ZCU Details}
\label{ZCUDetails}

\subsection{General Configuration}
The ZCU102 has been configured to run Petalinux 2023.2 without modifications to the kernel or user applications. There is a modified device tree blob (DTB) that includes a modified Si570 frequency of 156.25MHz (per XES/IEEE specification) and multiple instantiations within the amba\_pl block for configuring the necessary clocks of the EthHelper. Additionally, there is a reserved memory region of 256MB in DRAM for testing purposes to avoid data collisions. 

\subsection{Memory Alignment and Caching}
The Orchestrator is capable of handling unaligned accesses without issue due to it being more of a passthrough/monitor module on the critical route. Simply, all transactions, given that components upstream and downstream of the Orchestrator work correctly together. However, the Cortex-A53, by default, is not configured to allow unaligned accesses in device memory or non-cacheable memory. Yet, this feature of unaligned access has been supported since ARMv6\cite{ARMv6M_Architecture_Reference_Manual}. This is corrected in two possible ways: modifications to the Memory Attribute Indirection Register (MAIR), or making the memory region cacheable. We chose for the latter as a colleague had a module from \cite{Izhbirdeev2024} that could be modified with ease with memremap() to allocate a 3MB range at the specific memory apertures of the FPGA module.
%how in-depth should I go? 

%https://docs.amd.com/r/en-US/ug1085-zynq-ultrascale-trm/100G-Ethernet

%https://docs.amd.com/r/en-US/ug1085-zynq-ultrascale-trm/Interlaken

%https://docs.amd.com/r/en-US/ug1085-zynq-ultrascale-trm/GTH-and-GTY-Transceivers

%things I can mention 
% additional hardware support for the 8B/10B, 64B/66B, or 64B/67B encoding schemes to provide a sufficient number of transitions. this refers to what actually attaches to the sfp+ cages
\section{FPGA In-depth view}
%go over zcu102 FPGA connections and resources

\subsection{\axifull~ to \axistream~ translation}
%Highlight the problem with serializing 5 channels and the FSM involved to make that happen
% To comply with the available Xilinx Ethernet subsystem module would necessitate serializing the multiple channels of \axifull~
The Xilinx Ethernet Submodule (XES) used for this preliminary study has the datapath for RX/TX built as an \axistream~ interface. Along with the limitation of 4 physical ports for the XES to utilize, there required a serializer to  convert the 5 channels of \axifull~ into a single \axistream~ channel. This responsibility formed into the Orchestrator module.

The module itself is quite simple and is composed of a simple round-robin scheduler and a finite state machine (make reference and image) internally with output signals to actuate sub-modules, which the end user could create, that handled the interface responsibilities.  Each submodule is responsible for a single channel on an interface. The submodules send up information of valid and ready signals, in progress status, transaction length, along with metadata and data to the Orchrestrator. The Orchestrator's passes down resets, clock, and \axistream~ ready signals to the submodules. The Orchestrator uses a set of encoding masks that allow the proper control for all modules at in a single clock cycle, so that the proper submodule can unblock a pending transaction. The encoding mask design choice was made to retain the lowest channel transaction latency possible. The Orchestrator was also made with the intent of an easily modifiable (offline or online) and understandable wrapper and framework for any serialization to \axistream~. 

Given a simple assumption that the downstream \axistream~ is always ready. We can formulate a simple equation for the longest wait time that one module needs to wait to unblock. Given $N$ enabled submodules and a function $Burst\_Length()$ which gives the transaction length for the  $n\_i$  we can make ~\ref*{eq:orchestrator_latency}:
\begin{equation}
    \sum_{n=1}^N 2*Burst\_length(n_i)+2*N
    \label{eq:orchestrator_latency}
\end{equation}

Where the longest latency will be for the last submodule as it will have to wait the orchestrator to send all metadata and data from each burst of each submodule, with the added overhead of enabling and receiving data from each submodule.


\subsection{Frame Former}
%Lowest latency means I have to send packets the second I get info and decouple the last signals
This module allows the decoupling of signals and configuring data link layer parameters. The FrameFormer comprises a FrameFormerSubordinate (FFS) and the FrameFormerManager (FFM). 

The FFS submodule provides a shifting register array (that is user-defined in length) to buffer incoming data from the Orchestrator in order to minimize the amount of blocking the downstream sending mechanisms present. The blocking may come from frame-forming actions (e.g., sending start of frame and sending the end of frame) to physical module or protocol actions (inter-frame gap wait, resets on error, etc). 
The initial motivation for using a shift register was to have the latest data on a wire and minimize the logic to refresh the data. In comparison, a traditional approach would have an additional register to store the output, whereas we intended the first register of the shifting register array to uphold this responsibility. This choice allowed all logic to revolve around the single object of the shifting register. It may have only had a slight clock difference. %Still, part of implementing this was a learning experience on how to initially code in SystemVerilog and rework the execution idea from lines of code to clock-based actuation.

FFM is a simpler module that executes the actual framing of a packet and sends the correct signal to the XES for proper transmission. Furthermore, this module allows the users (and in the future programs) to select the parameters of the ethernet frame (e.g. source/destination address, ethernet type, and frame length) to other parameters such as sync words (i.e. to show start and end of FPGA stream) via configuration ports. 

\subsection{10g/25g Ethernet Subsystem}
%IEEE Std 802.3ba compliant physical interface for 100gb/s
To facilitate testing the Xilinx 10g/25g Ethernet module was employed for availability and documentation purposes. It does require an additional license for xxv-ethernet-3.1 which comes free of cost yet is only available for 180 days. 
The module has an \axifull~ interface for configuration of the MANY options. However, the default module configuration is ready to transmit (given the correct signaling) data. Furthermore, there are two \axistream~ interfaces for rx/tx. 



%Worst Negative slack 1.751ns 
%Worst Hold Slack 0.01ns
%Worst Pulse Width slack 0.304ns
%design was synthesizable with the Orchestrator at 300MHz but did not work correctly when switching Petalinux versions


